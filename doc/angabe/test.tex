\documentclass[a4paper, 11pt]{article}
\usepackage{comment} % enables the use of multi-line comments (\ifx \fi) 
\usepackage{lipsum} %This package just generates Lorem Ipsum filler text. 
\usepackage{fullpage} % changes the margin

\usepackage{graphicx}
\usepackage{wrapfig}
\usepackage{sidecap}
\usepackage[ngerman]{babel}
\usepackage{hyperref}
\usepackage{textcomp}
\usepackage{caption}
\usepackage{subcaption}

\graphicspath{{Figures/}}

\begin{document}

\noindent
\large\textbf{ITP} \hfill \textbf{Chuck Norris} \\
\normalsize WS 2024/25 \hfill 1.Test/1ITP2/\\
Prof. Aberger/Reder \hfill Datum: 16.11.2024\\
\\

\abstract{}
Es ist eine Anwendung zu erstellen, die Chuck Norris Programmierer Witze anzeigt.
Für den Zugriff und die Suche von Witzen gibt es ein vorgegebenes REST - Backend.
Für dieses Backend ist eine Web-Anwendung zu erstellen, mit der Witze geladen werden können.
Die Datenbank ist relativ umfangreich. Deshalb lädt die Anwendung niemals alle Witze in den Browser,
sondern es wird immer nur ein einziger zufälliger Witz aus der Datenbank geladen.

\section{Frameworks}

Die Anwendung soll gut langzeit - wartbar sein, deshalb werden nur folgende Frameworks verwendet:
\begin{itemize}
	\item lit-html
\end{itemize}


\section{Screenshot}

In der Abbildung \ref{fig:img} sehen Sie, wie eine solche Anwendung live aussehen soll. Es sind keine Seiten - Reloads erlaubt und
es gibt kein Flackern beim Aktualisieren von Witzen.

\section{User Interface}
Das User Interface besteht aus einem Reload - Button der nur per REST einen neuen zufälligen Witz abfragt. Dabei findet kein vollständiger Seiten-Neuaufbau statt.

\section{Server, Backend und Frontend}
Die Datenbank wird gestartet mit

\begin{verbatim}
	startmysql
\end{verbatim}

Username und Passwort finden sie in \begin{verbatim}backend/src/main/resources/application.properties\end{verbatim}

Ein Startprojekt steht am backend - Verzeichnis bereit. 
Der Server wird gestartet mit:
\begin{verbatim}
	cd ./backend
	mvn quarkus:dev
\end{verbatim}
Danach steht das REST - API zur Verfügung unter:
\href{http://localhost:8080}{http://localhost:8080}

Eine Beispielanwendung mit allen Dependencies finden Sie im ./frontend Unterverzeichnis.
Wechseln Sie in dieses Verzeichnis und starten Sie dann mit
\begin{verbatim}
npm start
\end{verbatim}

dann steht das REST-API auch über den Proxy 
\href{http://localhost:4200/api}{http://localhost:4200/api} zur Verfügung\footnote{dies ist wegen CORS eher zu empfehlen als der 8080 Port}.

In der Datei ./api/requests.http finden Beispiele für alle REST - Aufrufe, die Sie benötigen. 
\section{Anweisungen}

Bei der Abgabe vorher fdie node modules und backend/target/ löschen. Die Abgabe erfolgt als .zip Datei.


\section{Punkteverteilung}
Maximal können 100 Punkte erreicht werden. Die Punkteverteilung ist in Tabelle \ref{tab:points} angegeben.
\begin{table}[ht]
\centering
\caption{Punkteverteilung}

\begin{tabular}{| l | r |} \cline{1-2} 
\textbf{Aufgabe} &  \textbf{Punkte}  \\ \cline{1-2}
Backend liefert Witze (siehe api/requests.http) & 35 \\ \cline{1-2}
Anzeige eines Witzes im Browser & 25 \\ \cline{1-2}
Bei Reload nur einen winzigen Witz aus der DB laden & 18 \\ \cline{1-2}
Der Witz ist zufällig & 10 \\ \cline{1-2}
Reload ohne Seiten-Neuaufbau & 10 \\ \cline{1-2}
Abgabe laut \textit{Anweisungen} korrekt & 2 \\ \cline{1-2}
\end{tabular}
\label{tab:points}
\end{table}

\begin{figure}
	\center
	\includegraphics[width=\textwidth]{chuck-joke}
	\caption{Screenshot}
	\label{fig:img}
\end{figure}

\end{document}
