\documentclass[a4paper, 11pt]{article}
\usepackage{comment} % enables the use of multi-line comments (\ifx \fi) 
\usepackage{lipsum} %This package just generates Lorem Ipsum filler text. 
\usepackage{fullpage} % changes the margin

\usepackage{graphicx}
\usepackage{wrapfig}
\usepackage{sidecap}
\usepackage[ngerman]{babel}
\usepackage{hyperref}
\usepackage{textcomp}
\usepackage{caption}
\usepackage{subcaption}

\graphicspath{{Figures/}}

\begin{document}

\noindent
\large\textbf{Webtechnlogie} \hfill \textbf{Pokèmon} \\
\normalsize WS 2024/25 \hfill 1.Test/1ITP2/\\
Prof. Aberger/Reder \hfill Datum: 16.11.2024\\
\\

\abstract{}
Es ist eine Anwendung zu erstellen, die Pokemons in Tabellenform auflisten kann.
Für den Zugriff und die Suche von Pokèmons gibt es ein vorgegebenes REST - Backend.
Für dieses Backend ist eine Web-Anwendung zu erstellen, die einen Überblick über die angebotenen Produkte gibt.
Die Datenbank ist relativ umfangreich. Deshalb lädt die Anwendung niemals alle Items in den Browser,
sondern es wird immer nur eine Seite geladen mit typischeweise 10 Items. Dies gilt auch für die Suche.

\section{Frameworks}

Die Anwendung soll gut langzeit - wartbar sein, deshalb werden nur folgende Frameworks verwendet:
\begin{itemize}
	\item RxJs
\end{itemize}


\section{Screenshot}

Im testall Verzeichnis finden Sie die Datei pokemon.mp4. Dort sehen Sie, wie eine solche Anwendung live aussehen soll. Es sind keine Seiten - Reloads erlaubt und
es gibt kein Flackern beim Aktualisieren von Daten.

\section{User Interface}
Für die Suche ist oberhalb ein Suchfeld.\footnote{Jeder Tastendruck löst eine neue Suche aus.}
Unterhalb der Tabelle befinden sich Buttons zum Vor- und zurückblättern, sowie Buttons die 1/10 der Gesamtseitenanzahl vor- oder zurückspringen
und Buttons die an den Anfang bzw das Ende springen. Dies gilt auch für den Fall, dass mehr Suchergebnisse 
vorhanden sind als auf eine Seite passen.

\section{Server}
Ein Startprojekt pokemon.tgz steht am testall Laufwerk zum Download bereit. 
Der Server wird gestartet mit:
\begin{verbatim}
	cd ./server
	npm start
\end{verbatim}
Danach steht das REST - API zur Verfügung unter:
\href{http://localhost:8080}{http://localhost:8080}

Eine Beispielanwendung mit allen Dependencies finden Sie im ./client Unterverzeichnis.
Wechseln Sie in dieses Verzeichnis und starten Sie dann mit
\begin{verbatim}
npm start
\end{verbatim}

dann steht das REST-API auch über den Proxy 
\href{http://localhost:4200/api}{http://localhost:4200/api} zur Verfügung\footnote{dies ist wegen CORS eher zu empfehlen als der 8080 Port}.

In der Datei ./api/api.http finden Beispiele für alle REST - Aufrufe, die Sie benötigen. 
\section{Anweisungen}

- Da die Suchtabelle auch für andere Projekte wieder verwendbar werden soll, muss sie als Custom Element mit Shadow-DOM imlementiert werden.
- Die Datei index.html ist schreibgeschützt und darf nicht verändert werden\footnote{diese wird beim Korrigieren der Arbeit mit der Originaldatei überschrieben}.
- Es ist darauf zu achten, dass die Seite nicht flackert. 
- Zu lange Texte werden abgeschnitten und werden sichtbar, wenn man mit der Maus darüber hovered.
\section{Unterlagen}

Am TestAll - Laufwerk ist die Dokumentation der Funktionalität von Standard Web-Browsern zu finden (developer.mozilla.org.tgz).
Diese kann auf den lokalen Rechner kopiert und geöffnet werden. Ausserdem finden Sie einige andere Unterlagen dort.

\section{Punkteverteilung}
Maximal können 100 Punkte erreicht werden. Die Punkteverteilung ist in Tabelle \ref{tab:points} angegeben.
\begin{table}[ht]
\centering
\caption{Punkteverteilung}

\begin{tabular}{| l | r |} \cline{1-2} 
\textbf{Aufgabe} &  \textbf{Punkte}  \\ \cline{1-2}
Anzeige der Pokèmons in der Tabelle & 30 \\ \cline{1-2}
Anzeige der Bilder(Sprite) zum Produkt & 10 \\ \cline{1-2}
Seitenweise Darstellung (nächste/vorige Seite) & 15 \\ \cline{1-2}
vollstänige Pagination (vor/zurück/start/ende/grosse Schritte)& 10\\ \cline{1-2}
Suche & 25\\ \cline{1-2}
Seitenweise Darstellung bei Suche & 10\\ \cline{1-2}
\end{tabular}
\label{tab:points}
\end{table}

\begin{figure}
	\center
	\includegraphics[width=\textwidth]{pokemon}
	\label{fig:img}
\end{figure}

\end{document}
